\chapter{Conclusion And Future Work}
\label{chapter:conclusion} 
Over the past decade, machine learning has been applied to various safety-critical domains, and it's crucial for classifiers to be interpretable, fair, robust, and private to ensure trustworthy and responsible AI. In this thesis, we focus on the interpretability and fairness aspects of machine learning and aim to improve the scalability and accuracy of the underlying problems. We utilize formal methods to make the following contributions: (i) In interpretable machine learning, we design an incremental learning technique for interpretable rule-based classifiers of varied expressiveness. (ii) In fairness in machine learning, we develop a formal probabilistic fairness verification framework that can verify multiple fairness definitions of classifiers.  Additionally, we develop techniques to interpret fairness metrics by identifying feature combinations responsible for the bias of the classifier.



To demonstrate the efficacy of our methods, we have constructed open-source tools and conducted experiments on real-world datasets in machine learning. In the context of interpretable rule-based machine learning, we have developed an incremental learning framework, known as {\imli}, that scales classification to million-size datasets while maintaining competitive prediction accuracy and rule size compared to existing rule-based classifiers. Additionally, in our pursuit of more expressive yet interpretable classifiers, we have introduced another learning framework, called {\crr}, for logical relaxed classification rules that are based on incremental learning. Experimental results show that {\crr} is capable of learning more concise and accurate rule-based classifiers.


Our work on fairness in machine learning introduces two novel probabilistic fairness verifiers, {\justicia} and {\fvgm}, which exhibit superior performance in accuracy and scalability compared to state-of-the-art verifiers. {\justicia} is a stochastic-SAT-based verifier that enables scalable verification of fairness for compound sensitive groups of Boolean classifiers, such as decision trees, which was previously infeasible with existing methods. On the other hand, {\fvgm} takes correlated features into account and is capable of verifying the fairness of linear classifiers with higher scalability and accuracy than previous verifiers. Additionally, we discuss a global sensitivity analysis-based method, {\fairXplainer}, that interprets group fairness metrics by computing fairness influences of individual and intersectional features. Notably, {\fairXplainer} approximates bias more accurately using fairness influence functions (FIFs) and demonstrates a higher correlation of FIFs with fairness intervention than the local interpretability-based approach.







Our future research is dedicated to developing practical and scalable algorithms for trustworthy machine learning. Machine learning and artificial intelligence have been compared to the new electricity, with the potential to transform various aspects of human life, evident from the overwhelming response to generative AI. Ensuring fairness and interpretability in deployed machine learning is now more necessary than ever. To accomplish this, we aim to work in a collaborative environment, gaining insights into real-world challenges and leveraging advances in the field, alongside formal methods, to make significant progress. We have identified key research themes that will guide our work towards this vision.


\paragraph{Fairness and Interpretability As a Service.} 
The goal of modern machine learning extends beyond learning patterns from large-scale historical data to ensuring responsible decision-making through careful regulation to establish trustworthiness. For instance, in a job application scenario, a machine learning algorithm must be fair across different demographic groups, resilient to non-actionable changes in candidate profiles, and interpretable to allow candidates to understand the decision-making process. The long-term research goal of this thesis is to offer fairness and interpretability as a service with machine learning-based decision-making. 

\begin{itemize}
	
	
	\item \textbf{Interpretability with Guarantees.} Our research in interpretable machine learning spans \textbf{two-fold directions}.  (i) \textit{Interpretability by design:} There is a growing interest for interpretable machine learning in safety-critical domains~\cite{rudin2019stop}. Building upon our interpretable rule-based classifier $ \mathsf{IMLI} $, we aim to enhance learning algorithms for interpretable models in large-scale datasets across supervised, semi-supervised, and unsupervised settings. (ii) \textit{Post-hoc interpretability:} To explain black-box predictions, we focus on explanations with formal guarantees~\cite{ignatiev2019abduction,ribeiro2018anchors,guidotti2018local}. For example, an explanation model must be robust, learned in a privacy-preserving manner, and provide the confidence level of explanations to increase transparency and trust in the decision-making process.
	
	
	
		\item \textbf{Fairness Auditing.} Any technology that is publicly used presently undergoes an audit mechanism, where we understand the impacts and limitations of using that technology, and why are they caused. Machine learning is becoming the pervasive technology of our time and the discourse on bias induced by machine learning systems is attracting attention, e.g. the demonstration of bias in popular generative machine learning and large language models~\cite{abid2021persistent,nadeem2020stereoset,vig2020investigating}. As a result, there has been significant research interest in auditing classifiers for bias, from standard supervised learning to deep neural networks, computer vision, large language models and so on. Thus, we aim to design a fairness auditing framework~\cite{ruf2021towards,yan2022active} with formal guarantees.  There are three key questions that we aim to investigate in fairness auditing.
	
	\begin{enumerate}
		\item \textbf{Which fairness metrics to choose?} Fairness in machine learning is bestowed with multiple notions of fairness. Our first line of investigation is to categorize different fairness metrics  and suggest the best metric based on application, data, and prevailing policy, similar to ``Fairness Compass''~\cite{ruf2021towards}. This would help stakeholders pick the right definition of fairness for their application.
		
		\item \textbf{How to quantify bias?} Accurate quantification of bias is an important step towards designing algorithms to mitigate bias. As discussed in the thesis, fairness verification allows us to formally quantify the bias of a classifier. To this end, we aim to extend formal fairness verification to broader classes of fairness metrics, classifiers, and data. 
		
		
		\textbf{Fairness Metrics.} We aim to extend fairness verification of beyond group fairness, such as individual fairness~\cite{john2020verifying}, causal fairness~\cite{pan2021explaining,zhang2018fairness}, and counterfactual fairness~\cite{wu2019counterfactual,chiappa2019path}. Each category of fairness metrics poses distinct challenges in formal verification. For example, verifying individual fairness is related to verifying robustness of a model. As such, statistical methods from robustness verification has been applied to individual fairness~\cite{john2020verifying}. Formal methods, such as SMT-based encoding is also proposed in this regard~\cite{biswas2022fairify}. In our research endeavor, we aim to leverage  SAT or quantified Boolean formula (QBF) based verification, which may improve the scalability of verification.
		
		
		\textbf{Broader Classifiers.} We aim to extend fairness verification to broader machine learning classifiers, such as random forests and deep neural networks. For random forests, we can leverage CNF-based translation of the ensemble of trees by converting each tree as a CNF and a cardinality constraint to implement the ranking function of the prediction of multiple trees. For a special case of binarized neural networks (BNNs)~\cite{hubara2016binarized}, we can leverage existing CNF encoding and deploy SSAT-based verifier. However, for general neural networks with continuous parameters, MILP-based encoding can possibly be explored~\cite{mistry2022milp}. In addition, for neural networks, other surrogate representation beyond CNF or MILP can also be studied in future. In all cases, fairness verification, particularly for group-based metrics, relies on an access to efficient counter of CNF/MILP encoding; thus a dedicated effort to design better counter is a challenging, yet important research direction to explore.
		
		
		\textbf{Complex data.} We have explored fairness verification of tabular data; an important research question is to formally verify the fairness of classifiers with complex data such as images~\cite{nuriel2021permuted} and languages~\cite{abid2021persistent,nadeem2020stereoset,vig2020investigating}. For such data, one way to adapt existing verification methods, such as {\justicia}, is to apply them on the learned feature representation by the neural network and propagate verification results back to the input layer with images or text. In future, we aim to explore this possibility.
		
		\item \textbf{How to explain bias?} We aim to design fair and interpretable algorithms for  machine learning. We are interested in how these two goals relate to each other. This is inline with GDPR's emphasis on making models transparent and trustworthy, which is deployed in public.	Towards bridging the gap between fairness and interpretability in predictive systems, we explore following research questions:
		
		\begin{enumerate}
			\item How fair are interpretable machine learning models? 
			\item How can we improve the fairness of interpretable models?
			\item How can we apply interpretability to enhance fairness?
			\item How can we jointly optimize a classifier for  fairness and interpretability?
		\end{enumerate}
	
	
	\end{enumerate}
\end{itemize}



\paragraph{Verifiable Machine Learning with Formal Methods.} In safety-critical and high-stake domains, the verification of machine learning before deployment is crucial. To this end, formal methods provide a template to verify different checkable properties of machine learning. In particular, SAT, SMT and their variants allow to concentrate on synthesizing constraints from real-world use cases and delegate the solution finding to respective solvers, thanks to the dedicated community in formal methods. Building upon our ongoing research, which involves applying SSAT and MaxSAT to address fairness and interpretability challenges, we plan to explore additional formulations in formal methods, such as functional analysis, abstract interpretation, and solvers with expressive theory, to further enhance the verification of machine learning models.






\paragraph{Benchmarking Machine Learning: Improvement in Formal Methods.} Interdisciplinary research is an ongoing process that yields valuable contributions to different disciplines. Our work on fairness and interpretability in machine learning has had a positive impact on the improvement of solvers in formal methods. For instance, when tackling interpretable classification problems, we realized that MaxSAT alone was insufficient for large-scale rule-based classification, prompting the need for an incremental MaxSAT solver. As a result, we contribute to the MaxSAT evaluation competition in 2019 with our MaxSAT benchmarks of interpretable classification. Subsequently, the competition introduced an incremental MaxSAT solving track. This observation serves as strong motivation for us to continue contributing to the formal methods community by creating additional machine learning verification benchmarks.




