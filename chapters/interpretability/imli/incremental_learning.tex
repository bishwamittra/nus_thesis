\section{Incremental Learning with MaxSAT-based formulation}
\label{interpretability_imli_sec:incremental_learning}
The complexity of the MaxSAT query in Section~\ref{interpretability_imli_sec:encoding} increases with the number of samples in the training dataset and the number of clauses to be learned in a CNF formula. In this section, we discuss an incremental learning technique that along with the MaxSAT formulation can generate classification rules efficiently. Our incremental learning is built on two concepts: (i)
mini-batch learning and (ii) iterative learning. We discuss both in detail in the following.

\subsection{Mini-batch Learning} 

Our first improvement is to implement a mini-batch learning technique tailored for rule-based classifiers. Mini-batch learning has two-fold advantages. Firstly, instead of solving a large MaxSAT query for the whole training data, this approach solves a sequence of smaller MaxSAT queries derived from mini-batches of smaller size.  \blue{Secondly, this approach extends to online learning, where the classifier can be updated incrementally with new samples while also generalizing to previously observed samples.} In our context of rule-based classifiers, we consider the following heuristic in mini-batch learning.


\paragraph{A Heuristic for Mini-Batch Learning.} Let $ \Rule' $ be a classifier learned on the previous batch. In mini-batch learning, we aim to learn a new classifier $ \Rule $ that can generalize to both the current batch and previously seen samples. For that, we consider a soft constraint such that $ \Rule $ does not \emph{differ much} from $ \Rule' $ while training on the current batch. Thus, we hypothesize that by constraining $ \Rule $ to be syntactically similar to $ \Rule' $, it is possible to generalize well; because samples in all batches originate from the same distribution. Since our study focuses on rule-based classifiers, we consider the Hamming distance between two classifiers as a notion of their syntactic dissimilarity. In the following, we define the Hamming distance between two CNF classifiers $ \Rule $, $ \Rule' $ with the same number of clauses as follows.
\[
	d_{H}(\Rule, \Rule') = \sum_{i=1}^k \Big(\sum_{v\in C_i} \mathds{1}(v \not\in C'_i) +  \sum_{v\in C'_i} \mathds{1}(v \not\in C_i) \Big), 
\]
 
 where $ C_i $ and $ C'_i $ are the $ i^\text{th} $ clause in $ \Rule $ and $ \Rule' $, respectively and $ \mathds{1} $ is an indicator function that returns $ 1 $ if the input is true and returns $ 0 $ otherwise.  Intuitively, $ d_{H}(\Rule', \Rule) $ calculates the total number of different literals in each (ordered) pair of clauses in $ \Rule $  and $ \Rule' $.  For example, consider $ \Rule = (X_1 \vee X_2) \wedge (\neg X_1) $ and $ \Rule' = (\neg X_1 \vee X_2) \wedge (\neg X_1) $. Then $ 	d_{H}(\Rule, \Rule') = 2 + 0 = 2 $, because in the first clause, the  literals in $ \{X_1, \neg X_1\} $ are absent in either formulas, and the second clause is identical for both $ \Rule $ and $ \Rule' $. In the following, we discuss a modified objective function for mini-batch learning.



\paragraph{Objective Function.}
\label{interpretability_imli_sec:obj_incremental}
In mini-batch learning, we design an objective function to penalize both classification errors on the current batch and the Hamming distance between new and previous classifiers. Let $ \dataset_b \subset \dataset $ be the current mini-batch, where $ |\dataset_b| \ll |\dataset| $. The objective function in mini-batch learning is


%\red{Moreover, $ (\red{\dataset}, \mathbf{y}) $ is the training dataset, on which we measure the performance of $ \Rule $ after completing incremental learning\textemdash we will expand shortly on how we measure the performance. 
%}


\begin{align}
\label{interpretability_imli_eq:obj_incr}
\min_{\Rule} |\mathcal{E}_{\dataset_b}|+\lambda d_{H}(\Rule, \Rule').
\end{align}

In the objective function, $ \mathcal{E}_{\dataset_b} $ is the misclassified subset of samples in the current batch $ \dataset_b $. Unlike controlling the rule-sparsity in the non-incremental approach in the earlier section, in Eq.~\eqref{interpretability_imli_eq:obj_incr}, $ \lambda $ \emph{controls the trade-off between classification errors and the syntactic differences between consecutive classifiers}. Next, we discuss how to encode the modified objective function as a MaxSAT query. 
 

\paragraph{MaxSAT Encoding of Mini-batch Learning.}
\label{interpretability_imli_sec:encoding_incremental}
In order to account for the modified objective function, we only modify the soft clause $ S_j^i $ in the MaxSAT query $ Q $. In particular, we define $ S_j^i $ to penalize for the complemented assignment  of the feature variable $ \bool_j^i $ in $ \Rule $ compared to $ \Rule' $.
\[
S_j^i:=
\begin{cases}
\bool_j^i& \text{if }  X_j \in  \clause(\Rule',i)  \\
\neg \bool_j^i & \text{otherwise}
\end{cases};\qquad \clauseweight{S_j^i}=\lambda
\]

$ S_j^i $ is either a unit clause $ \bool_j^i $ or $ \neg \bool_j^i $ depending on whether the associated feature $ X_j $ appears in the previous classifier $ \Rule' $ or not. In this encoding, the Hamming distance of $ \Rule $ and $ \Rule' $ is minimized while attaining minimum classification errors on the current batch (using soft clause $ E_l $ in Eq.~\eqref{interpretability_imli_soft_clause_error}) To this end,  mini-batch learning starts with an empty CNF formula as $ \Rule' $ without any feature, and thus $ S_j^i := \neg \bool_j^i $ for the first batch. We next analyze the complexity of the MaxSAT encoding for mini-batch learning. 


\begin{proposition}
	\label{interpretability_imli_prop:maxsat_complexity_incremental}
	Let $ n' \triangleq |\dataset_b| \ll |\dataset| $ be the size of a mini-batch, $ n'_{neg} \le n' $ be the number of negative samples in the batch, and $ m $ be the number of Boolean features. According to Proposition~\ref{interpretability_imli_prop:maxsat_variables}, to learn a $ k $-clause CNF classifier in mini-batch learning, the MaxSAT encoding for each batch has $ km + n' $ Boolean variables and $ kn'_{neg} $ auxiliary variables. Let $ n'_{pos} = n' - n'_{neg} $ be the number of positive samples in the batch. Then, according to Proposition~\ref{interpretability_imli_prop:maxsat_clauses}, the number of clauses in the MaxSAT query for each batch is $ k m+n'+k n'_{pos}+(k m+1)n'_{neg} \approx \mathcal{O}(k m  n' ) $ when $ n'_{pos}= n'_{neg} =\frac{n'}{2} $.
\end{proposition}



%
%\paragraph{Constructing Mini-batch.}
%\red{Can be put in experiments.}
%We construct mini-batches in several ways: (i) randomly sampling a fixed number of samples from the training data, and (ii) sequentially partitioning the training data into batches of fixed size. In our experiments, we have focused on the later approach. Additionally, to ignore the effect of batch-order, we repeat the mini-batch learning multiple times.

\paragraph{Assessing  the Performance of $ \Rule $.} Since we apply a heuristic objective in mini-batch learning,  $ \Rule $ may be optimized for the current batch while generalizing poorly on the full training data. To tackle this, after learning on each batch, we decide whether to keep $ \Rule $ or not by assessing the performance of $ \Rule $ on the training data and keep $ \Rule $ if it achieves higher performance. We measure the performance of $ \Rule $ on the full training data $ \dataset $ using a weighted combination of classification errors and rule-size. In particular, we compute a combined loss function $ \mathsf{loss}(\Rule) \triangleq |\mathcal{E}_\dataset| +\lambda |\mathcal{R}| $ on the training data $ \dataset $, which is indeed the value of the objective function that we minimize in the non-incremental learning in Section~\ref{interpretability_imli_sec:problem}. Additionally, we  discard the current classifier $ \Rule $ when the loss does not decrease ($ \mathsf{loss}(\Rule) > \mathsf{loss}(\Rule') $). 

We present the algorithm for mini-batch learning in Algorithm~\ref{interpretability_imli_algo:incremental}. 
\begin{algorithm}
	\caption{MaxSAT-based Mini-batch Learning}
	\label{interpretability_imli_algo:incremental}
	\begin{algorithmic}[1]
		\Function{MiniBatchLearning}{$ \dataset, \lambda, k $}
		\State $\mathcal{R} = \leftarrow \bigwedge_{i=1}^k\text{false} $ \Comment{Empty CNF formula}
		\State $ \mathsf{loss}_{\max} \leftarrow \infty$ 
		\For{$ i \leftarrow 1, \dots, N $} \Comment{$ N $ is the total batch-count}
		\State $ \dataset_b \leftarrow \textsc{GetBatch}(\dataset) $
		\State $ Q, \clauseweight{\cdot} \leftarrow \textsc{MaxSATEncoding}(\Rule, \dataset_b, \lambda, k) $ \Comment{Returns a weighted-partial CNF}
		\State $ \sigma^* \leftarrow \MaxSAT(Q,\clauseweight{\cdot}) $
		\State $ \Rule_{\mathsf{new}} \leftarrow \textsc{ConstructClassifier}(\sigma^*) $
		\State $ \mathsf{loss} \leftarrow  |\mathcal{E}_\dataset| +\lambda |\mathcal{R}_{\mathsf{new}}| $ \Comment{Compute loss}
		\If{$ \mathsf{loss} < \mathsf{loss}_{\max} $}
		\State $ \mathsf{loss}_{\max} \leftarrow \mathsf{loss} $
		\State $ \Rule \leftarrow \Rule_{\mathsf{new}} $
		\EndIf
		\EndFor
		\State \Return $ \Rule $
		\EndFunction
	\end{algorithmic}
\end{algorithm} 	

