\section{Interpretable Classification Rule Learning via MaxSAT}
\label{interpretability_imli_sec:baseline}
In this section, we revisit~\cite{malioutov2018mlic} by discussing a MaxSAT-based learning framework for an interpretable rule-based classifier, particularly a CNF classifier $ \Rule $. 
We first describe the decision variables in Chapter~\ref{interpretability_imli_sec:variables} and present the MaxSAT encoding  in Chapter~\ref{interpretability_imli_sec:encoding}. The MaxSAT formulation assumes binary features as input. We
conclude this section by learning $ \Rule $ with non-binary features in Chapter~\ref{interpretability_imli_sec:non-binary} and discussing more flexible interpretability constraints of $ \Rule $ in Chapter~\ref{interpretability_imli_sec:complex_interpretability_objectives}.  



\subsection{Description of Variables}
\label{interpretability_imli_sec:variables} 
We initially preprocess feature vector $ \mathbf{x} $  to account for the negation of  Boolean features while learning a classifier.\footnote{This preprocessing is similarly applied in  \cite{malioutov2013exact}.} In the preprocessing step, we negate each feature in $ \mathbf{x} $ to a new feature and append it to $ \mathbf{x} $. For example, if ``$ \text{age }\ge 25 $'' is a Boolean feature, we add another feature ``$ \text{age }< 25 $'' in  $ \mathbf{x} $ by negating the feature ``$ \text{age }\ge 25 $''. Hence, in the rest of the chapter, we refer $ m $ as the modified number of features in $ \mathbf{x} $. We next discuss the variables in the MaxSAT problem. 

We consider two types of Boolean variables: (i) \emph{feature} variables $ \bool $ corresponding to input features and (ii) \emph{error} variables $ \noise $ corresponding to the classification error of samples. We define a Boolean variable $ \bool^i_j $ that becomes true if feature $ X_j $ appears in the $ i^\text{th} $ clause of $ \Rule $, thereby contributing to an increase in the rule-size of $ \Rule $, and $ \bool^i_j $ is assigned false, otherwise. Moreover, we define an error variable $ \noise_l $ to attribute to whether the $ l^\text{th} $ sample $ (\mathbf{x}^{(l)}, y^{(l)}) $ is classified correctly or not. Specifically, $ \noise_l $ becomes true if $ (\mathbf{x}^{(l)}, y^{(l)}) $ is misclassified, and becomes false otherwise.  We next discuss the MaxSAT encoding to solve the classification problem.   




\subsection{MaxSAT Encoding}
\label{interpretability_imli_sec:encoding}
We consider a partial-weighted MaxSAT formula, where we encode the objective function in Eq.~\eqref{interpretability_imli_eq:obj} as soft clauses and the learning constraints as hard clauses. We next discuss the MaxSAT encoding in detail.

\begin{itemize}
	\item \textbf{Soft clauses for maximizing training accuracy:} For each training sample, we construct a soft unit\footnote{A unit clause has a single literal.} clause $ \neg \noise_l $ to account for a penalty for misclassification. Since the penalty for misclassification of a sample is $ 1 $  in Eq.~\eqref{interpretability_imli_eq:obj},  the weight of this soft clause is also $ 1 $. 
	
	\begin{equation}
		\label{interpretability_imli_soft_clause_error}
		E_l:=\neg \noise_l;  \qquad \clauseweight{E_l}= 1
	\end{equation}
	Intuitively, if a sample is misclassified, the associated error variable becomes true, thereby dissatisfying the soft clause $ E_l $.
	 
	
	\item \textbf{Soft clauses for minimizing rule-sparsity:} To favor rule-sparsity, we learn a classifier with as few literals as possible. Hence, for each feature variable $ \bool_j^i $, we construct a unit clause as $ \neg \bool_j^i $. Similar to training accuracy, the weight for this clause is derived as $ \lambda $ from Eq.~\eqref{interpretability_imli_eq:obj}.
	\begin{equation}
		\label{interpretability_imli_soft_clause_feature}
		S_j^i:= \neg \bool_j^i; \qquad \clauseweight{S_j^i}= \lambda
	\end{equation}
	

	
	\item \textbf{Hard clauses for encoding constraints:} In a MaxSAT problem, constraints that must be satisfied are encoded as hard clauses. In rule-based classification, we have a learning constraint that, if the error variable is false, the associated sample must be correctly classified, and vice-versa. Let  $\boolset_i= \{\bool^i_{j} \mid j \in \{1,\dots,m \}\}$ be a vector of feature variables corresponding to the $ i^\text{th} $  clause in $ \mathcal{R} $. Then, we define the following hard clause.
	
	\begin{equation}
			\label{interpretability_imli_hard_clause_rule_consistency}
		H_l:= \neg \noise_l \rightarrow \Big(y^{(l)}\leftrightarrow \bigwedge_{i=1}^k {\mathbf{x}^{(l)}} \circ {\boolset_{i}}\Big); \qquad  \clauseweight{H_l}=\infty
	\end{equation}
	
	In the hard clause, $ \bigwedge_{i=1}^k {\mathbf{x}^{(l)}} \circ {\boolset_{i}}\ $ is a CNF formula including variables $ \bool_j^i $ for which the associated  feature-value is $ 1 $ in  $ \mathbf{x}^{(l)} $. Since $ y^{(l)} \in \{0,1\} $ is a constant, the constraint to the right of the implication ``$ \rightarrow $'' is either  $ \bigwedge_{i=1}^k {\mathbf{x}^{(l)}} \circ {\boolset_{i}}\ $ or its complement. Therefore, the hard clause enforces that if the sample is correctly classified (using $ \neg \noise_l $), either  $ \bigwedge_{i=1}^k {\mathbf{x}^{(l)}} \circ {\boolset_{i}}\ $ or its complement is true depending on the class-label of the sample.  We highlight that  the single-implication ``$ \rightarrow $'' in the hard clause $ H_l $ acts as a double-implication ``$ \leftrightarrow $''  due to the soft clause $ E_l $. Because, according to the definition of ``$ \rightarrow $'', the left constraint $ \neg \noise_l $ can  be false while the right constraint of 	``$ \rightarrow $'' is true.  This, however, incurs unnecessarily dissatisfying the soft clause $ E_l $, which is a sub-optimal solution and hence this solution is not returned by the MaxSAT solver. 
	
%	We highlight that according to the definition of ``$ \rightarrow $'', the left constraint $ \neg \noise_l $ can  be false while the right constraint of 	``$ \rightarrow $'' is true.  This, however, incurs unnecessarily dissatisfying the soft clause $ E_l $, which is a sub-optimal solution and hence this solution is not returned by the MaxSAT solver. Therefore, the single-implication ``$ \rightarrow $'' in the hard clause $ H_l $ acts as a double-implication ``$ \leftrightarrow $''  due to the soft clause $ E_l $. 
	

	


\end{itemize}

	  We next discuss the translation of soft and hard clauses into a CNF formula, which can be invoked by any MaxSAT solver. 

\paragraph{Translating $ E_l, S^i_j, H_l $ to a CNF formula.} 
The soft clauses $ E_l $ and $ S_j^i $ are unit clauses and hence, no translation is required for them.  In the hard clause, when $ y^{(l)}=1 $, the simplification is $ H_l:= \neg \noise_l \rightarrow \bigwedge_{i=1}^k {\mathbf{x}^{(l)}} \circ {\boolset_{i}}  $. In this case, we apply the equivalence rule in propositional logic $ (A \rightarrow B) \equiv (\neg A \vee B) $  to encode $ H_l $ into  CNF. In contrast, when $ y_i=0 $, we simplify the hard clause as $ H_l:= \neg \noise_l \rightarrow \neg (\bigwedge_{i=1}^k {\mathbf{x}^{(l)}} \circ {\boolset_{i}}) \Rightarrow  \neg \noise_l \rightarrow \bigvee_{i=1}^k \neg({\mathbf{x}^{(l)}} \circ {\boolset_{i}})$. Since $ \mathbf{x}^{(l)} \circ \boolset_{i} $ constitutes a disjunction of literals, we apply Tseytin transformation to encode  $ \neg (\mathbf{x}^{(l)} \circ \boolset_{i}) $ into CNF. More specifically, we introduce an auxiliary variable $ z_{l,i} $ corresponding to the clause $ \neg ({\mathbf{x}^{(l)}} \circ {\boolset_{i}}) $. Formally, we replace $ H_l := \neg \noise_l \rightarrow  \bigvee_{i=1}^k \neg ({\mathbf{x}^{(l)}} \circ {\boolset_{i}}) $ with   $ \bigwedge_{i=0}^k H_{l,i} $,  where $ H_{l,0}:= (\neg \noise_l \rightarrow  \bigvee_{i=1}^k  z_{l,i}) $  and  $  H_{l,i}:= z_{l,i} \rightarrow  \neg({\mathbf{x}^{(l)}} \circ {\boolset_{i}})  $ for $ i=\{1,\dots,k\} $. Finally, we  apply the equivalence of $ (A \rightarrow B) \equiv (\neg A \vee B) $ on $ H_{l,i} $ to translate them into CNF. For either value of $ y^{(l)} $, the weight of each translated hard clause in the CNF formula is $ \infty $.

% Now,  the hard clause can be simplified as $ H_l:= \neg \noise_l \rightarrow \bigwedge_{i=1}^k {\mathbf{x}^{(l)}} \circ {\boolset_{i}}  $ when $ y^{(l)}=1 $, otherwise  $ H_l:= \neg \noise_l \rightarrow \neg (\bigwedge_{i=1}^k {\mathbf{x}^{(l)}} \circ {\boolset_{i}}) \Rightarrow  \neg \noise_l \rightarrow \bigvee_{i=1}^k \neg({\mathbf{x}^{(l)}} \circ {\boolset_{i}})$. 



% In the hard clause,  $\bigwedge_{i=1}^k {\mathbf{x}^{(l)}} \circ {\boolset_{i}}  $ is already in CNF. When $ y^{(l)}=1 $, the constraint $ H_l $ can be directly encoded as CNF by using equivalence of $ (a \rightarrow b) \equiv (\neg a \vee b) $. When $ y_i=0 $, we apply 



Once we translate  all soft and hard clauses into CNF, the MaxSAT query $ Q $ is the conjunction of all clauses. 
\[
Q:= \bigwedge_{l=1}^n E_l \wedge  \bigwedge_{i=1,j=1}^{i=k,j=m} S_j^i \wedge \bigwedge_{l=1}^n H_l 
\]
		
Any off-the-shelf MaxSAT solver can output an optimal assignment $ \sigma^* $ of the MaxSAT query $ (Q, \clauseweight{\cdot}) $. We extract $ \sigma^* $ to construct the classifier $ \Rule $ and compute training errors as follows.
\begin{construction}
	\label{interpretability_imli_construction:rule}
	Let $\sigma^* = \MaxSAT(Q,\clauseweight{\cdot})$. Then $X_j \in {\clause(\Rule,i)}$ if and only if $\sigma^*(\bool_{j}^{i}) = 1$. Additionally, $ (\mathbf{x}^{(l)}, y^{(l)}) $ is misclassified if and only if $ \sigma^*(\noise_l) = 1 $.
\end{construction}




	
	
	 
\paragraph{Complexity of the MaxSAT query.}

We analyze the complexity of the MaxSAT query in terms of the number of Boolean variables and clauses in the CNF formula $ Q $. 

\begin{proposition}
	\label{interpretability_imli_prop:maxsat_variables}
	 To learn a $ k $-clause CNF classifier for a dataset of  $ n $ samples over $ m $ boolean features, the MaxSAT query $ Q $ defines $ km +n $ Boolean variables. Let $ n_{neg} $ be the number of negative samples in the training dataset. Then the number of auxiliary variables in $ Q $ is  $ kn_{neg}  $.
\end{proposition}


\begin{proposition}
	\label{interpretability_imli_prop:maxsat_clauses}
	The MaxSAT query $ Q $ has $ k m+n $ unit clauses corresponding to the constraints $ E_l $ and $ S_j^i $. For each positive sample, the hard clause $ H_l $ is translated into $ k $  clauses. For each negative sample, the CNF translation requires at most $ k m+1 $ clauses. Let $ n_{pos} $ and $ n_{neg} $ be the number of positive and negative samples in the training set. Therefore, the number of clauses in the MaxSAT Query $ Q $ is $ k m+n+k n_{pos}+(k m+1)n_{neg} \approx \mathcal{O}(k m  n ) $ when $ n_{pos}= n_{neg} =\frac{n}{2} $. 
\end{proposition}

\begin{example}{MaxSAT Encoding.}
\normalfont

We illustrate the MaxSAT encoding for a toy example consisting of four samples and two binary features. Our goal is to learn a two clause CNF classifier that can approximate the training data. 
\[ 
\dataset_\mathsf{orig} = 
\begin{blockarray}{ccc}
X_1 & X_2 & Y\\
\begin{block}{[ccc]}
0 & 0 & 1\\
0 & 1 & 0\\
1 & 0 & 0\\
1 & 1 & 1\\
\end{block}
\end{blockarray} \implies 
\dataset = 
\begin{blockarray}{ccccc}
X_1 & \neg X_1 & X_2  & \neg X_2 & Y\\
\begin{block}{[ccccc]}
0 & 1 & 0 & 1 & 1\\
0 & 1 & 1 & 0 & 0\\
1 & 0 & 0 & 1 & 0\\
1 & 0 & 1 & 0 & 1\\
\end{block}
\end{blockarray}
\]
In the preprocessing step, we negate the features $ \{X_1, X_2\} $ in $ \dataset_\mathsf{orig} $ and add complemented features $ \{\neg X_1, \neg X_2\} $ in $ \dataset $. In the MaxSAT encoding, we define $ 8 $ feature variables ($ 4 $ features $ \times $ $ 2 $ clauses in the classifier) and $ 4 $ error variables. For example, for feature $ X_2 $, introduced feature variables are $ \{\bool^1_3, \bool^2_3\} $ and for feature $ \neg X_2 $, introduced variables are $ \{\bool^1_4, \bool^2_4\} $. For four samples, error variables are $ \{\noise_1, \noise_2, \noise_3, \noise_4\} $. We next show the soft and hard clauses in the MaxSAT encoding


\[
\begin{split}
&E_1:= (\neg \noise_1); \quad
E_2:= (\neg \noise_2); \quad E_3:= (\neg \noise_3); \quad
E_4:= (\neg \noise_4)\\
& S_1^1 := (\neg \bool_1^1);\quad 
 S^1_2 := (\neg \bool^1_2);\quad 
 S^1_3 := (\neg \bool^1_3);\quad
 S^1_4 := (\neg \bool^1_4);\quad \\ 
& S_1^2 := (\neg \bool_1^2);\quad 
S^2_2 := (\neg \bool^2_2);\quad 
S^2_3 := (\neg \bool^2_3);\quad 
S^2_4 := (\neg \bool^2_4);\quad\\ 
& H_1:= (\neg \noise_1 \rightarrow ((\bool_2^1 \vee \bool_4^1)\wedge (\bool_2^2 \vee \bool_4^2)));\\
& H_2:= (\neg \noise_2 \rightarrow (\neg(\bool_2^1 \vee \bool_3^1) \vee \neg(\bool_2^2 \vee \bool_3^2)));\\
& H_3:= (\neg \noise_3 \rightarrow (\neg (\bool_1^1 \vee \bool_4^1) \vee \neg(\bool_1^2 \vee \bool_4^2)));\\
& H_4:= (\neg \noise_4 \rightarrow ((\bool_1^1 \vee \bool_3^1) \wedge (\bool_1^2 \vee \bool_3^2)));\\
\end{split}
\]

In this example, we consider regularizer $ \lambda = 0.1 $, thereby setting the weight on accuracy as $ \clauseweight{E_i} = 1 $ and rule-sparsity weight as $ \clauseweight{S_j^i} = 0.1 $. Intuitively, the penalty for misclassifying a sample is $ 10 $ times than the penalty for adding a feature in the classifier. For this MaxSAT query, the optimal solution classifies all samples correctly by assigning four feature variables $ \{\bool^1_1, \bool^1_4, \bool^2_2, \bool^2_3\} $ to true. Therefore, by applying Construction~\ref{interpretability_imli_construction:rule}, the classifier is $ (X_1 \vee \neg X_2)  \wedge (\neg X_1 \vee X_2) $.\footnote{The presented MaxSAT-based formulation does not learn a CNF classifier with both $ X_i $ and $ \neg X_i $ in the same clause. The reason is that a clause with both $ X_i $ and $ \neg X_i $ connected by OR $ (\vee) $ does not increase accuracy but increases rule-size and hence, this is not an optimal classifier.}
	
\end{example}
	
	
\subsection{Learning with Non-binary Features}
	\label{interpretability_imli_sec:non-binary}
	The presented MaxSAT encoding requires input samples to have binary features. Therefore, we discretize datasets with categorical and continuous features into binary features. For each continuous feature, we apply equal-width discretization that splits the feature into a fixed number of bins. For example, consider a continuous feature $ X_c \in [a,b] $. In discretization, we split the domain $ [a,b] $ into three bins with two split points $ \{a',b'\} $ such that $ a<a'<b'<b $.  Therefore, the resulting three discretized features are $ { a \le X_c < a'} $, $ {a' \le X_c < b'} $, and $ { b' \le X_c  \le b } $. An alternate to this close-interval discretization is open-interval discretization, where we consider six discretized features with each feature being compared with one threshold. In that case, the discretized features are: $ X_c \ge a, X_c \ge a', X_c \ge b', X_c < a', X_c < b', X_c \le b $. Both open-interval and close-interval discretization techniques have their use-cases where one or the other is appropriate. For simplicity, we experiment with close-interval discretization in this chapter.
	
	After applying discretization on continuous features, the dataset contains categorical features only, which we convert to binary features using one-hot encoding~\cite{lakkaraju2019faithful}. In one-hot encoding, a Boolean vector of features is introduced with cardinality equal to the number of distinct categories. For example, consider a categorical feature having three categories `red', `green', and `yellow'. In one hot encoding, samples with category-value `red', `green', and `yellow' would be converted into binary features by taking values $ 100 $, $ 010 $, and $ 001 $, respectively. 
	
	
\subsection{Flexible Interpretability Objectives}
\label{interpretability_imli_sec:complex_interpretability_objectives}
In interpretable classification rules, we can consider more flexible interpretability objectives than the simplified one in Eq.~\eqref{interpretability_imli_eq:obj}. In Eq.~\eqref{interpretability_imli_eq:obj}, we prioritize all features equally by providing the same weight to the clause $ S_j^i $ for all  values of $ i $ and $ j $. In practice, users may prefer rules containing certain features. In the MaxSAT formulation, such an extension can be achieved by modifying the weight function and/or the definition of  $ S_j^i $. For example, to constrain the classifier to never include a feature, the weight of the clause $ S_j^i := \neg \bool_j^i $ can be set to $ \infty $. In contrast, to always include a feature, we can define $ S_j^i := \bool_j^i $ with weight $ \infty $. In both cases, we treat $ S_j^i $ as a hard clause. 

Another use case may be to learn rules where clauses have disjoint set of features. To this end, we consider a pseudo-Boolean constraint $ \sum_{i=1}^k \bool^i_j \le 1 $, which specifies that the feature $ X_j $ appears in at-most one of the $ k $ clauses. This constraint may be soft or hard depending on the priority in the application domain. In either case, we convert this constraint to CNF using pseudo-Boolean to CNF translation~\cite{philipp2015pblib}. Thus, the MaxSAT formulation allows us to consider varied interpretability constraints by only modifying the MaxSAT query without requiring changes in the MaxSAT solving. Therefore, the \emph{separation between modeling and solving} turns out to be the key strength of the presented MaxSAT formulation.



