\chapter{Introduction}
The last decades have witnessed significant progress in machine learning with a host of applications of algorithmic decision-making in different safety-critical domains, such as medical~\cite{erickson2017machine,kaissis2020secure,kononenko2001machine}, law~\cite{kumar2018law,surden2014machine}, education~\cite{luckin2018machine}, and transportation~\cite{peled2019model,zantalis2019review}. In high-stake domains,  machine learning predictions have far-reaching consequences on the end-users~\cite{eshete2021making}. With the aim of applying machine learning for societal goods, there have been increasing efforts to regulate machine learning by imposing interpretability~\cite{rudin2019stop}, fairness~\cite{barocas2017fairness}, robustness~\cite{rauber2017foolbox}, and privacy~\cite{papernot2016towards} in predictions. In this thesis, we focus on the interpretability and fairness aspects of machine learning. The thesis establishes a close integration of formal methods and automated reasoning with machine learning and proposes efficient algorithmic solutions for problems arising in interpretability and fairness in machine learning.

Towards responsible and trustworthy machine learning, we propose two research themes in this thesis: interpretability and fairness of machine learning classifiers. \emph{In interpretable machine learning}, rule-based classifiers effectively represent the decision boundary using a set of rules comprising input features. Interpretable rule-based classifiers not only interpret the decision function but also are applied to explain the prediction of black-box classifiers~\cite{gill2020responsible,lundberg2017unified,moradi2021post,ribeiro2016should,slack2020fooling}, a fundamental research question in explainable artificial intelligence (XAI). In this thesis, we propose efficient algorithms based on incremental learning for interpretable rule-based classifiers. In another research theme of \emph{fairness in machine learning}, unregulated classifiers tend to exhibit bias/unfairness to certain demographic groups in the data unless classifiers are trained with a fairness objective. Consequently, research on fairness centers on quantifying bias using multiple fairness definitions and mitigating bias based on multiple fairness algorithms. In this thesis, we study probabilistic fairness verification problem, where we formally verify the bias of a classifier given the probability distribution of input features. Finally, we combine both research themes: interpretability and fairness and propose a framework to interpret the sources of bias. In particular, we formalize and compute fairness influence functions, a way to quantify the influence of individual and the intersection of multiple features on the bias of a classifier. To summarize our thesis on interpretability and fairness in machine learning, we prioritize on improving the \emph{scalability} and the \emph{accuracy} of solutions\textemdash either or both of which are absent in prior works. 




\section{Interpretable Rule-based Machine Learning}
The problem in interpretable machine learning is to learn a classifier making interpretable predictions to the end-users. To achieve the interpretability of predictions, decision functions in the form of classification rules such as decision trees,  decision lists, decision sets, etc.\ are particularly effective~\cite{bessiere2009minimising,dash2021lprules,ignatiev2021reasoning,izza2020explaining,lakkaraju2017interpretable,lakkaraju2016interpretable,letham2015interpretable,narodytska2018learning,rivest1987learning,wang2015falling,yu2020optimal}.  At this point, it is important to acknowledge that interpretability is an informal notion, and formalizing it in full generality is challenging. In our context of rule-based classifiers, we use \emph{sparsity of rules} (that is, fewer rules each having fewer Boolean literals), which has been considered a proxy of interpretability in various domains, specifically in the medical domain~\cite{gage2001validation,lakkaraju2019faithful,letham2015interpretable,malioutov2013exact,myers1962myers}.





In the thesis, we study two interpretable rule-based classifiers, characterized by their expressiveness. At first, we study classifiers represented as formulas in propositional logic. In propositional logic, Conjunctive Formal Form (CNF) and Disjunctive Normal Form (DNF) are useful representations of Boolean formulas. Popular interpretable rule-based classifiers such as decision tree, decision lists, and decision sets share the logical structure of CNF/DNF in their representation of the decision function. Thereby, we propose to learn an interpretable CNF classifier, wherein its interpretability is defined by the number of Boolean literals that the formula contains. Compared to CNF, Boolean cardinality constraints are more expressive as they allow numerical bounds on Boolean literals~\cite{sinz2005towards}. Relying on the concept of cardinality constraints to increase expressiveness, our second interpretable rule-based classifier is a logical relaxation of CNF/DNF classifiers, namely \emph{relaxed-}CNF. 


The problem of learning rule-based classifiers is known to be computationally intractable. The earliest tractable approaches for classifiers such as decision trees and decision lists relied on heuristically chosen objective functions and greedy algorithmic techniques~\cite{ClarkN1989,CohenS1999,quinlan2014}. In these approaches, the size of rules is controlled by early stopping,  ad-hoc rule pruning, etc. In recent approaches, the interpretable classification problem is reduced to an optimization problem, where the accuracy and sparsity of rules are optimized jointly~\cite{lakkaraju2016interpretable,narodytska2018learning}. Different optimization solvers such as linear programming~\cite{malioutov2013exact}, sub-modular optimizations~\cite{lakkaraju2016interpretable}, Bayesian optimizations~\cite{letham2015interpretable}, and MaxSAT~\cite{malioutov2018mlic} are then deployed to find the best classifier with maximum accuracy and minimum rule-size. The discrete combinatorial nature of learning rule-based classifiers leads to the intractability of the problem and suffers from scalability issues in large datasets. Therefore, we propose an incremental learning approach by wrapping traditional optimization solvers such as MaxSAT and MILP to efficiently learn rule-based classifiers in a mini-batch learning setting.  Here we summarize the contributions of the thesis on interpretable rule-based machine learning.

\subsection{Scalability via Incremental Learning}
We propose an incremental learning framework, called $ \imli $,  based on MaxSAT for learning interpretable classification rules expressed in CNF. $ \mathsf{IMLI} $ considers a joint objective function to optimize the accuracy and sparsity of classification rules and learns a rule-based classifier by solving an appropriately designed MaxSAT query. Despite the progress of MaxSAT solving in the last decade, the straightforward MaxSAT-based solution cannot scale to real-world classification datasets of thousands to millions of samples. Therefore, we incorporate an efficient incremental learning technique inside the MaxSAT formulation by integrating mini-batch learning and iterative rule-learning. The resulting framework learns a classifier by iteratively covering the training data, wherein in each iteration, it solves a sequence of smaller MaxSAT queries corresponding to each mini-batch. In our experiments, $ \mathsf{IMLI} $ achieves the best balance among prediction accuracy, interpretability, and scalability. For instance, $ \mathsf{IMLI} $ attains a competitive prediction accuracy and interpretability with respect to existing interpretable classifiers and demonstrates impressive scalability on large datasets where both interpretable and non-interpretable classifiers fail. As an application, we deploy $ \mathsf{IMLI} $ in learning other interpretable representations: DNF classifiers, decision lists, and decision sets.
	
\subsection{Expressiveness via Logical Relaxation}
We extend our incremental learning framework to learn a more relaxed representation of classification rules with higher expressiveness. Elaborately, we consider relaxed definitions of standard OR/AND operators in propositional logic by allowing exceptions in the construction of a clause and also in the selection of clauses in a rule. Building on these relaxed definitions, we introduce relaxed-CNF classification rules motivated by the popular usage of checklists in the medical domain and the higher expressiveness of Boolean cardinality constraints. Relaxed-CNF generalizes widely employed rule representations including CNF, DNF, and decision sets. While the combinatorial structure of relaxed-CNF rules offers exponential succinctness, the na\"ive learning techniques are computationally expensive. To this end, we propose an incremental mini-batch learning procedure, called $ \mathsf{CRR} $, that employs advances in MILP solvers to efficiently learn relaxed-CNF rules. Our experimental analysis demonstrates that $ \mathsf{CRR} $ can generate relaxed-CNF rules, which are more accurate and sparser compared to the alternative rule-based models.

\textbf{Tool:} We publish a python library \href{https://github.com/meelgroup/MLIC}{{\imli}} for learning interpretable rule-based classification rules in a supervised learning setting.







\section{Fairness in Machine Learning}
As a technology machine learning is oblivious to societal good or bad. The success of machine learning as an accurate predictor, however, finds applications in high-stake decision-making, such as college admission~\cite{martinez2021using}, recidivism prediction~\cite{tollenaar2013method}, job applications~\cite{ajunwa2016hiring} etc. In such applications, the deployed  classifier often demonstrates bias towards certain \emph{sensitive demographic groups} involved in the data~\cite{dwork2012fairness}. For example, a classifier deciding the eligibility of college admission may offer more admission to White-male candidates than to Black-female candidates\textemdash possibly because of the historical bias in the admission data, or the accuracy-centric learning objective of the classifier, or a combination of both~\cite{berk2019accuracy,landy1978correlates,zliobaite2015relation}. Following such phenomena, multiple fairness metrics, such as \textit{statistical parity}, \textit{equalized odds}, \textit{predictive parity} etc, have been proposed to quantify the bias of the classifier. For example, if the classifier in college admission demonstrates a {statistical parity} of $ 0.6 $, it means that White-male candidates are offered admission $ 60\% $ more frequently than Black-female candidates~\cite{besse2021survey,feldman2015certifying,garg2020fairness}. To this end, different fairness enhancing algorithms have been devised to improve fairness with respect to one or multiple fairness metrics. These algorithms try to rectify and mitigate bias in three ways: \textit{pre-processing} the data~\cite{kamiran2012data,zemel2013learning,calmon2017optimized}, \textit{in-processing} the classifier~\cite{zhang2018mitigating}, and \textit{post-processing} the outcomes of a classifier~\cite{kamiran2012decision,hardt2016equality}. There is also study on fairness attack algorithms to worsen the fairness of a classifier, such as by adding poisoned data samples~\cite{solans2020poisoning}. In the presence of multiple fairness metrics and algorithms, in this thesis, we contribute to two fundamental problems in fairness: (i) probabilistic verification of fairness and (ii) interpreting sources of unfairness.


\subsection{Probabilistic Fairness Verification} The problem in probabilistic fairness verification is to verify the bias of a classifier given the distribution of input features. The early works on fairness verification focused on measuring fairness metrics of a classifier for a given dataset~\cite{aif360-oct-2018}. Naturally, such techniques were limited in enhancing confidence of users for wide deployment. Consequently, recent verifiers seek to achieve verification beyond  finite dataset and in turn focus on the  probability distribution of features~\cite{albarghouthi2017fairsquare, bastani2019probabilistic}.  More specifically, the input to the verifier is a classifier and  the probability distribution of features, and the output is an estimate of fairness metrics that the classifier obtains given the distribution.


In order to solve the fairness verification problem, existing works have proposed two principled approaches.	Firstly,~\cite{albarghouthi2017fairsquare} propose a formal method approach to reduce the verification problem into the weighted volume computation of an SMT formula. Secondly,~\cite{bastani2019probabilistic} propose a sampling approach that relies on extensively enumerating the conditional probabilities of prediction given different sensitive features and thus, incurs high computational cost. Additionally, existing works assume feature independence of non-sensitive features and consider correlated features within a limited scope, such as conditional probabilities of non-sensitive features w.r.t.\ sensitive features and ignore correlations among non-sensitive features. As a result, the \textit{scalability} and \textit{accuracy} of existing  verifiers remain major challenges.


\subsubsection{Formal Fairness Verification} In this thesis, we propose an efficient fairness verification framework, starting with a general approach for finite classifiers by encoding them as Boolean formulas, and later a special case of linear classifiers. Based on stochastic satisfiability (SSAT)\cite{littman2001stochastic}, our proposed verifier {\justicia} verifies the fairness of Boolean classifiers such as decision tree by solving appropriately designed SSAT formulas. {\justicia} also extends verification to compound sensitive groups, which are a combination of multiple categorical sensitive features such as race $ \in $ \{White, Black\} and gender $ \in $ \{male, female\}. SSAT encoding, by construction, allows separate quantification to each sensitive feature without any restriction on the number of features, thereby it is natural in SSAT-based formulation to extend the verification problem to multiple sensitive features unlike earlier methods. In experiments, {\justicia} is more scalable than the existing probabilistic verifiers~\cite{albarghouthi2017fairsquare,bastani2019probabilistic} because of the efficient SSAT encoding.
%, and more robust than the sample-based empirical verifiers~\cite{aif360-oct-2018}. 
%We also prove a finite-sample error bound on estimated fairness metrics, which is stronger than the existing asymptotic guarantees.


\subsubsection{Tractable Fairness Verification With Feature Correlation} Linear classifiers have attracted significant attention from researchers in the context of designing and testing prototype fairness enhancing and attack algorithms~\cite{pleiss2017fairness,zafar2017fairness,dressel2018accuracy, john2020verifying}. In the context of verifying the bias of linear classifiers, existing fairness verifiers suffer from two-fold limitations: (i) poor scalability due to applying SSAT/SMT or sampling based techniques and (ii) inaccuracy due to ignoring feature correlations. To alleviate these limitations, we propose a fairness verification framework for linear classifiers, namely {\fvgm}, for an accurate and scalable fairness verification. {\fvgm} proposes a novel \textit{stochastic subset-sum} encoding for linear classifiers with an efficient pseudo-polynomial solution using dynamic programming. To address feature-correlations, {\fvgm} considers a graphical model, particularly a Bayesian Network to represent the conditional dependence (and independence) among features in the form of a Directed Acyclic Graph (DAG). Experimentally,  {\fvgm} is more accurate and scalable than existing fairness verifiers; {\fvgm} can verify group and causal fairness metrics. We also demonstrate two novel applications of {\fvgm} as a fairness verifier: (a) detecting fairness attacks, and (b) computing fairness influences of a subset of features on shifting the incurred bias of the classifiers from the original bias.
%, which we study in detail in the following contribution of the thesis.


\textbf{Tool:} We publish a python library \href{https://github.com/meelgroup/justicia}{\justicia} for probabilistic fairness verification.



\subsection{Interpreting Fairness: Identifying Sources of Bias}
Fairness metrics measure global bias, but do not detect or interpret its sources~\cite{begley2020explainability,lundberg2020explaining,pan2021explaining}.
In order to diagnose the emergence of bias in the predictions of classifier, it is important to compute explanations, such as how different features attribute to the global bias. Motivated by the GDPR's ``right to explanation'', research on interpreting model predictions~\cite{ribeiro2016should,lundberg2017unified,lundberg2020local2global} has surged, but interpreting prediction bias has received less attention~\cite{begley2020explainability,lundberg2020explaining}. In order to identify and interpret the sources of bias and also the impact of affirmative/punitive actions to alleviate/deteriorate bias, it is important to understand \textit{which features contribute how much to the bias of a classifier} applied on a dataset. To this end, we follow a global feature-attribution approach to interpret the sources of bias, where we relate the \emph{influences} of input features towards the resulting bias of the classifier. In this context, existing bias attributing methods~\cite{begley2020explainability,lundberg2020explaining} are variants of local function approximation~\cite{sliwinski2019axiomatic}, whereas bias is a global statistical property of a classifier. Thus, \textit{we aim to design a bias attribution method that is global by construction}. In addition, existing methods only attribute the individual influence of features on bias while neglecting the \textit{intersectionality} among features. Quantifying intersectionality allows us to interpret bias induced by the higher-order interactions among features; hence accounting for intersectionality is important to understand bias as suggested by recent literature~\cite{buolamwini2018gender,wang2022towards}.  
\paragraph{Contribution.} We propose an algorithm, called {\fairXplainer}, based on global sensitivity analysis to estimate the fairness influence function (FIF) of individual and intersectional features. In experiments, {\fairXplainer} approximates bias based on FIFs more accurately than existing methods. We also demonstrate higher correlation of FIFs with fairness interventions, thereby demonstrating the importance of FIFs in designing improved fairness algorithms in future.

\textbf{Tool:} We publish a python library \href{https://github.com/ReAILe/bias-explainer}{\fairXplainer} for interpreting group-based fairness metrics in machine learning.

\section{Thesis Outline}
We organize the thesis in the following way. We discuss preliminaries in Chapter~\ref{chapter:preliminaries}. In the first part of the thesis on interpretable machine learning, we discuss an incremental learning framework based on MaxSAT for learning interpretable rule-based classifiers. We conclude this part by applying incremental learning on a more expressible yet interpretable rule-based classifier in Chapter~\ref{chapter:crr}. In the second part of the thesis on fairness in machine learning, we discuss probabilistic fairness verification for classifiers represented as Boolean formulas in Chapter~\ref{chapter:justicia} and linear classifiers in Chapter~\ref{chapter:fvgm}.  In the last part, we combine both interpretability and fairness in machine learning by interpreting group fairness metrics in Chapter~\ref{chapter:fif}. Finally, we conclude our thesis in Chapter~\ref{chapter:conclusion}.


\begin{comment}

The tremendous success of machine learning in recent years has found its applications in different high-stake and  safety-critical domains, such as medical, law, education, and transportation. In such domains, the interpretability and fairness of machine learning classifiers are of paramount importance. In this work, we focus on improving the interpretability and fairness in machine learning. 

There are two-fold objectives in interpretable machine learning: (i) designing classifiers that are interpretable by design and (ii) leveraging already interpretable models to explain the prediction of black-box classifiers. In this work, we propose an efficient learning framework~\cite{ghosh19incremental} for a rule-based classifier\textemdash a popular interpretable model because of expressing the decision function in terms of rules corresponding to input features.  Our framework allows us to learn classification rules expressible in propositional logic, for example, CNF (Conjunctive Normal Form) and DNF (Disjunctive formal form) formulas.  The benefit of learning CNF classifiers is that this is the fundamental block for learning other popular interpretable representations: decision lists and decision sets. Learning rule-based classifiers is an intractable problem because of solving a combinatorial optimization problem. To efficiently learn these classifiers, we propose a MaxSAT-based formulation and scale it by combining incremental learning. The result is an efficient learning framework that can classify datasets even with 1 million samples while also balancing between high accuracy and small rule size. As an application of incremental learning, we propose to learn more relaxed logical rules~\cite{ghosh20classification} which demonstrate higher succinctness in their representation while achieving better accuracy. Furthermore, as an application of MaxSAT, we solve group testing problems based on compressed sensing~\cite{ciampiconi20maxsat}. Our MaxSAT-based formulation for group testing has impressive scalability than state-of-the-art approaches. 



We study the application of interpretable machine learning models in explaining the prediction of black-box models such as neural networks. To this end, we propose a PAC-based (Probably Approximately Correct) local explanations of feed-forward and recurrent neural networks using rule-based classifiers and their variants~\cite{neider2020probably,ghosh2020formal}. To explain feed-forward neural networks, we learn flexible classification rules using SyGuS~\cite{neider2020probably}. Such rules have the theoretical property that they are PAC approximate to the neural network within the locality of a query of interest. Similarly, to explain recurrent neural networks, we propose learning explanations in LTL formulas (Linear Temporal Logic)~\cite{ghosh2020formal}. 


Machine learning models may become biased in their decisions towards certain demographic populations. One important question in fairness in machine learning is to verify whether the model is unfair or not with respect to multiple definitions of fairness studied in the literature. To this end, we propose a scalable fairness verification framework based on formal methods, particularly stochastic satisfiability (SSAT)~\cite{ghosh2020justicia} and stochastic subset-sum problem (S3P)~\cite{ghosh2021algorithmic}. The input to the fairness verifier is a machine learning model and the distribution of input features and the output is an assessment of different fairness metrics that the model obtains given the distribution. Our fairness verifier has the following advantages over existing verifiers. We can efficiently verify the fairness of compound sensitive groups\textemdash such as, white-male and black-female\textemdash by combining multiple sensitive features, unlike existing works that only consider a single binary sensitive feature. We also consider feature-correlation represented as a Bayesian network whereas existing methods consider limited correlation among features. As a result, the proposed verifier has higher accuracy than existing verifiers. Finally, our verifier is more scalable than existing verifiers because of the efficient formulation based on SSAT and S3P. 


As future work, we aim to combine the interpretability and fairness of machine learning in a single framework. In particular, we explain the unfairness of models by combining explanation methods in machine learning. Our hypothesis is that such a combination would be effective in identifying the bias of a model and also mitigate the bias through imperative actions.

\end{comment}

